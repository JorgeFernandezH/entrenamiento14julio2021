\documentclass[aspectratio=169,11pt]{beamer}


\title{Entrenamiento combinatoria $15$ de julio de $2021$ }
\date{}

\usetheme{Boadilla}
\usecolortheme{wolverine}

\begin{document}

% usaremos vietnam y rumania tsts

\frame{\titlepage}

\begin{frame}

	\begin{block}

		\begin{itemize}

				%romania tst 2021 day 2 p2
			\item Sea $M$ el conjunto $\{1,2,3,\dots,2020\}$. Encuentre el menor entero positivo $k$ tal que cualquier subconjunto $A$ de $M$ con $k$ elementos tiene tres elementos distintos $a,b,c$ tal que $a+b,b+c$ y $c+a$ están todos en $A$.

				%romania tst 2021 day 3 p2
			\item Sea $N\geq 4$ un entero positivo. Dos jugadores, Alicio y Bernarda están construyendo una secuencia, añadiendo elementos de forma alternada. En el primer turno Alicio puede añadir $1$ o $-1$, en el siguiente turno Bernarda puede añadir $2$ o $-2$, después Alicio puede añadir $3$ o $-3$ etc. El ganador es el primero en hacer que la suma sea un múltiplo de $N$. Encuentre quien es el ganador en términos de $N$.


		\end{itemize}

	\end{block}

\end{frame}

\begin{frame}

	\begin{block}

		\begin{itemize}

				%romania tst 2019 day 2 p2
			\item Determine el mayor entero positivo $N$ tal que cada arreglo de $5\times 5$ con los enteros del $1$ al $25$ (sin repetir) contiene un subtablero de $2\times2$ con suma el menos $N$.

				%romania tst 2019 day 4 p3
			\item Alicio y Bernarda juegan un juego. Inicialmente se coloca un montón de $i$ piedras sobre cada entero $i$ del $1$ al $n$. Inicialmente Alicio permuta los montones de alguna forma. Despues Bernardo selecciona un entero y quita una piedra de ese montón. En cada turno un jugador puede elegir un entero adyacente al anterior y retirar una piedra de ese montón. El primer jugador que no pueda hacer un movimiento pierde. Encuentre quien tiene la estrategia ganadora.


		\end{itemize}


	\end{block}

\end{frame}


\begin{frame}

	\begin{block}

		\begin{itemize}

				%romania tst 2015 day 5 p2
			\item  Se colorea las aristas de la gráfica $p$-partita completa $K_{n,n,\dots,n}$ con $p$ colores distintos. Para cada color $i$ sea $c_i$ el tamaño de la componenta conexa más grande cuando consideramos las aristas de color $i$. Encuentre el menor valor que puede tener $\max_{i=1}^n c_i$

				%romania tst 2012 day 3 p4
			\item Sea $S$ un conjunto de enteros positivos tal que cada uno tiene exactamento $100$ dígitos en base $10$. Decimos que un elemento de $S$ es atómico si no es divisible entre la suma de dos elementos (no necesariamente distintos) de $S$. Si $S$ contiene a lo más $10$ átomos, cual es la mayor cantidad de elementos que puede haber en $S$?

		\end{itemize}

	\end{block}

\end{frame}


\end{document}
