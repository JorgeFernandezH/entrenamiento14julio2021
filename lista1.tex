\documentclass[aspectratio=169,11pt]{beamer}


\title{Entrenamiento combinatoria $13$ de julio de $2021$ }
\date{}

\usetheme{Boadilla}
\usecolortheme{wolverine}

\begin{document}

% usaremos vietnam y rumania tsts

\frame{\titlepage}

\begin{frame}

	\begin{block}

		% euler cycle / brujin sequence
		\begin{itemize}
			\item	Se tiene una palabra de longitud $2021$ con las $26$ letras del alfabeto inglés. Definimos un error en la palabra como dos valores $i<j$ tal que los caracteres en posición $i$ y $j$ son iguales y los caracteres en posición $i+1$ y $j+1$ son iguales. ¿Cuál es el menor número de errores posible ?

				% decompose chess 4-cycles
			\item Los enteros no-negativos favoritos de Jaimico son $a$ y $b$. En un tablero de $12\times 12$ colocamos un número en cada cuadrito, de tal forma que el cuadro en la fila $i$ y columna $j$ tiene el número $ai + bj$. Colocamos algunos caballos de ajedrez en el tablero sin que haya dos que se ataquen. ¿Cuál es la mayor suma posible de los cuadritos ocupados?
		\end{itemize}


	\end{block}

\end{frame}

\begin{frame}

	\begin{itemize}

			\begin{block}

				%Vietnam tst 2021 p2
			\item En un tablero de $2021\times 2021$ seleccionamos $k$ cuadros unitarios tal que cada cuadro seleccionado comparte vértice con a lo más un cuadro seleccionado distinto. Determina el máximo valor posible de $k$.

				%Vietnam tst 2019 p1
			\item ¿Cuál es máximo número de colores con los que se puede colorear las aristas de $K_n$ tal que para cualquiera de los colores la gráfica con las aristas de esos colores ( y los $n$ vértices ) es conexa?


			\end{block}

	\end{itemize}


\end{frame}

\begin{frame}

	\begin{block}

		\begin{itemize}

				%Vietnam tst 2018 p2
			\item Decimos que un tablero de $m \times 2018$ en el que algunos cuadros tienen los dígitos $0$ y $1$ ( y algunos están vacíos) es completo si cada palabra de longitud $2018$ con caractéres $0$ y $1$ se puede obtener como una de las filas del tablero después de llenar algunos cuadros vacíos. Decimos que un tablero completo es minimal si al quitar cualquiera de sus filas deja de ser completo. A) Demuestre que para cada entero positivo $k\leq 2018$ existe un tablero minimal de tamaño $2^k\times 2018$ con $k$ columnas que contienen tanto $0$ como $1$. B) Suponga que un tablero minimal de tamaño $m\times 2018$ contiene exactamente $k$ columnas no-vacías. Demuestre que $m\leq 2^k$.

				%Vietnam tst 2018 p5
			\item Dados enteros positivos $n,m$ consideremos la gráfica con vértices $\{(a,b) | a\in [m+1], b\in [n+1]\}$ y donde $(a,b) \sim (x,y) \iff |a-x| + |b-y| = 1$. A) Demuestre que esta gráfica es hamiltoniana si y solo si al menos uno de $m$ y $n$ es impar. B) En caso de ser hamiltoniana encuentre el menor número de giros en un camino hamiltoniano de esta. Definimos un giro como un vértice $v$ tal que $v$ junto con su predecesor y sucesor en el ciclo no son colineales ( las parejas se pueden pensar como puntos en $\mathbb R^2$).

		\end{itemize}

	\end{block}

\end{frame}


\begin{frame}

	\begin{block}

		\begin{itemize}

				%vietnam tst 2012 p3
			\item Demueste que toda gráfica $20$-regular con $42$ vertices conexa tiene un emparejamiento perfecto.

		\end{itemize}

	\end{block}

\end{frame}


\end{document}
